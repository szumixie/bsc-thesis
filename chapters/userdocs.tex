\chapter{User documentation}\label{ch:userdocs}

Pirec is a functional language with dependent functions, dependent pairs, row
polymorphism and extensible records. It can be used to write programs without
side effect and prove simple theorems. The program is command line tool that can
type check and interpret Pirec programs.

\section{Installation}\label{sec:install}

The program can be installed as follows:
\begin{enumerate}
  \item Install \emph{Stack}~\cite{stack}.
  \item Clone or download the source code repository.
  \item Enter the repository and run the following command:
        \mintinline{text}{stack install}
\end{enumerate}

(During the compilation of \texttt{Paths\_pirec} module, there will be warning
about a prepositive qualified module. This is due to a bug in
\emph{Cabal}'s~\cite{cabal} automatic file generation in combination with the
warning which is only fixed in a recent pull request to the development version
of Cabal~\cite{cabal-prepos}.)

\section{Usage}\label{sec:usage}

A Pirec program can be written into a file with any source code editor.

Comments in Pirec are similar to Haskell and its derived languages, with
both line comments and block comments.
\begin{minted}{pirec.py:PirecLexer -x}
-- This is a line comment
{- This is a
   block comment -}
\end{minted}

The identity function can be defined like so:
\begin{minted}{pirec.py:PirecLexer -x}
id = λ (A : Type) → λ (a : A) → a
\end{minted}
It can then be applied to for example an empty record:
\begin{minted}{pirec.py:PirecLexer -x}
id (Rec #{}) rec{}
\end{minted}
where \mintinline{pirec.py:PirecLexer -x}{Rec #{}} is the type of empty records
and \mintinline{pirec.py:PirecLexer -x}{rec{}} is an instance of an empty
record. These two lines can be put into a file, for example into
\texttt{id.pirec}, then run \mintinline{text}{pirec id.pirec}. The program will
then output the following, prettyprinting the normal form of the last line of
the input and its inferred type:
\begin{minted}{pirec.py:PirecLexer -x}
rec{  }
  : Rec #{  }
\end{minted}
One can give command line options to the program to modify what to output, for
example \mintinline{text}{--elab-only} can be given so that the program does not
evaluate to normal form, instead it only elaborates and infers the type. Run
\mintinline{text}{pirec --help} to see all options.

Here is another way to define the identity function:
\begin{minted}{pirec.py:PirecLexer -x}
id2 : ∀ (A : Type) → A → A
    = λ _ a → a
\end{minted}
The definition is given a type signature, the lambda binders are grouped
together with a single lambda and their types are removed since they are
optional, the unused lambda parameter is ignored with an underscore. The
definition can also be split across multiple lines, as long as the following
lines have more indentation than the beginning of the definition

The first parameter can be made into an implicit parameter, and its type is also
optional:
\begin{minted}{pirec.py:PirecLexer -x}
id3 : ∀ {A} → A → A = λ a → a
\end{minted}
With this, the type of the argument does not need to be written out:
\begin{minted}{pirec.py:PirecLexer -x}
id3 rec{}
\end{minted}
Running \mintinline{text}{pirec id3.pirec --elab-only --metas} it outputs the
following:
\begin{minted}{pirec.py:PirecLexer -x}
let{ id3 = λ {A} a → a; id3 {?1} rec{  } }
  : Rec #{  }
Metavariables:
  ?0 = Type
  ?1 = Rec #{  }
\end{minted}
Here it shows that the implicit argument is elaborated to a metavariable which
is solved to the correct type.

Data types can be defined with their Church encoding (see \cref{ssec:church}),
for example the natural numbers:
\begin{minted}{pirec.py:PirecLexer -x}
Nat  : Type = ∀ {A} → (A → A) → A → A
zero : Nat  = λ _ z → z
suc  : Nat → Nat = λ x s z → s (x s z)
\end{minted}
One can then define operators on them like addition and multiplication:
\begin{minted}{pirec.py:PirecLexer -x}
add : Nat → Nat → Nat = λ x y s z → x s (y s z)
mul : Nat → Nat → Nat = λ x y s z → x (y s) z
\end{minted}
Then one can evaluate simple arithmetic expressions such as \(1 + 2 * 3\):
\begin{minted}{pirec.py:PirecLexer -x}
add (suc zero) (mul (suc (suc zero)) (suc (suc (suc zero))))
\end{minted}
Running this will get you the correct Church encoded number:
\begin{minted}{pirec.py:PirecLexer -x}
λ {A} s z → s (s (s (s (s (s (s z))))))
  : ∀ {A : Type} → (A → A) → A → A
\end{minted}

Records can be created in Pirec, for example:
\begin{minted}{pirec.py:PirecLexer -x}
r : Rec #{ x : Nat; y : Nat } = rec{ x = suc zero; y = suc (suc zero) }
\end{minted}
Record types are denoted with \mintinline{pirec.py:PirecLexer -x}{Rec}, then
follows a row, and \mintinline{pirec.py:PirecLexer -x}{#} is the start of a row
literal, following that are the two fields surrounded by braces which are both
\mintinline{pirec.py:PirecLexer -x}{Nat} in this case. Record literals are
denoted with the keyword \mintinline{pirec.py:PirecLexer -x}{rec}, following
that are the two fields which are given values. Records can be extended and
projected from, and restricted from:
\begin{minted}{pirec.py:PirecLexer -x}
r2 : Rec #{ x : Nat; z : Nat } = rec{ z = suc r.y | r.-y }
\end{minted}
Here there is a new \mintinline{pirec.py:PirecLexer -x}{z} field which takes the
same successor of the \mintinline{pirec.py:PirecLexer -x}{y} field, while the
original \mintinline{pirec.py:PirecLexer -x}{y} field is removed. Record
extension is like a literal, but with a pipe and the original record right of
it. Record projection and restriction are done with a period or
\mintinline{pirec.py:PirecLexer -x}{.-} followed by the field respectively.
There is syntactic sugar for modifying a field, which is the same as restricting then extending with the same field:
\begin{minted}{pirec.py:PirecLexer -x}
r3 : Rec #{ x : Nat; y : Nat } = rec{ x := zero | r.-x }
r4 : Rec #{ x : Nat; y : Nat } = rec{ x := zero | r }
\end{minted}
The above two records are the same, both have the
\mintinline{pirec.py:PirecLexer -x}{x} field become zero.

Row polymorphism can be used to manipulate records with unknown fields. For
example
\begin{minted}{pirec.py:PirecLexer -x}
f1 : ∀ {R} → Rec #{ x : Nat | R } → Rec #{ x : Nat | R }
   = λ r → rec{ suc r.x | r }
\end{minted}
is a function that takes any record with an \mintinline{pirec.py:PirecLexer
  -x}{x} field and modifies it to have the field have its successor. The
implicit argument \mintinline{pirec.py:PirecLexer -x}{R} is an arbitrary row,
and pipe is used for row extension similarly to record extension.

The extensible record is based on~\cite{scopedlabels}, so there multiple fields
can have the same name in the same row and record. For example
\begin{minted}{pirec.py:PirecLexer -x}
r5 : Rec #{ x : Nat; x : Nat } = rec{ x = suc zero; x = suc (suc zero) }
\end{minted}
is valid. Note that the order of fields with the matter when the name of the
fields are the same, in this case, projection and restriction only work on the
left most field of that name. To get a field which is not the left most, one can first restrict then project, like so:
\begin{minted}{pirec.py:PirecLexer -x}
r5.-x.x
\end{minted}

One can use dependent products and records to create a sort of list type. It is defined as follows:
\begin{minted}{pirec.py:PirecLexer -x}
List = λ A → ∃ (x : Nat) ×
               let z = #{}
                   s = λ (R : Row Type) → #{ e : A | R }
                   x s z
\end{minted}
It is a type for dependent pairs with a natural number, and a record with that
amount of fields all named \mintinline{pirec.py:PirecLexer -x}{e}. The dependent
product type is denoted similarly to the dependent function type, except that
\mintinline{pirec.py:PirecLexer -x}{∃} and \mintinline{pirec.py:PirecLexer
  -x}{×} are used instead. Right of the times sign, a let expression is used,
local definitions can be given with it with the last line as the return value.
The definitions and the return value must align, but there is an alternative
syntax, and can be like so:
\begin{minted}{pirec.py:PirecLexer -x}
List = λ A → ∃ (x : Nat) ×
               let{ z = #{}
                  ; s = λ (R : Row Type) → #{ e : A | R }
                  ; x s z }
\end{minted}
One might notice that the syntax of braces and semicolons is similar rows and
records. In fact, rows and records can also be defined without them. For example
here is an element of the \mintinline{pirec.py:PirecLexer -x}{List} type:
\begin{minted}{pirec.py:PirecLexer -x}
l : List Nat = suc (suc (suc zero)) ,
               rec e = zero
                   e = suc zero
                   e = suc (suc zero)
\end{minted}
The constructor of dependent pair is denoted with a comma.
\mintinline{pirec.py:PirecLexer -x}{Row} is the type of rows which can take any
type, but to use it with records it can only be \mintinline{pirec.py:PirecLexer
  -x}{Type}.

Every Unicode character shown has an \caps{ASCII} alternative which can be used instead, the correspondences are shown in \cref{tab:ascii}.

\newcommand\pirecampersand{\mintinline{pirec.py:PirecLexer -x}{&}}
\begin{table}
  \centering
  \begin{tabular}{@{}cc@{}}
    \toprule
    Unicode & \caps{ASCII} \\\midrule
    \mintinline{pirec.py:PirecLexer -x}{→}
            &
    \mintinline{pirec.py:PirecLexer -x}{->}
    \\
    \mintinline{pirec.py:PirecLexer -x}{∀}
            &
    \mintinline{pirec.py:PirecLexer -x}{forall}
    \\
    \mintinline{pirec.py:PirecLexer -x}{λ}
            &
    \mintinline{pirec.py:PirecLexer -x}{\ }
    \\
    \mintinline{pirec.py:PirecLexer -x}{×}
            &
    \pirecampersand{}
    \\
    \mintinline{pirec.py:PirecLexer -x}{∃}
            &
    \mintinline{pirec.py:PirecLexer -x}{exists}
    \\\bottomrule
  \end{tabular}
  \caption{\caps{ASCII} alternatives to Unicode syntax}\label{tab:ascii}
\end{table}

\section{Errors}\label{sec:errors}

\begin{figure}
  \begin{minted}{text}
examples/thousand.pirec:2:12:
  |
2 | y : Type = something x
  |            ^^^^^^^^^
variable something out of scope
  \end{minted}
  \caption{An example error}\label{fig:error-ex}
\end{figure}

Pirec can report many kinds of errors when given a program, it will also report
the position of the error in a visual way (see \cref{fig:error-ex}). The
following is a list of possible errors:
\begin{itemize}
  \item \begin{minted}{text}
pirec: ...: openFile: does not exist (No such file or directory)
        \end{minted}
        The file given in the command line argument does not exist.

  \item \begin{minted}{text}
unexpected ...
        \end{minted}
        A parse error when the parser was expecting some other token.

  \item \begin{minted}{text}
incorrect indentation ...
        \end{minted}
        A parse error when some token is at an incorrect indentation level. Due
        to the way parsing is implemented, sometimes this error occurs when an
        expression suddenly ends when it should not have.

  \item \begin{minted}{text}
variable ... out of scope
        \end{minted}
        A variable is referenced when there is no such variable in the current
        scope.

  \item \begin{minted}{text}
got ... application when ... application was expected
        \end{minted}
        An implicit argument was given to a function with explicit parameter or
        the other way around, and the program could not insert an implicit
        argument.

  \item \begin{minted}{text}
expected type:
  ...
but got inferred type:
  ...
        \end{minted}
        Could not match expected type and inferred type when unifying two types.

  \item \begin{minted}{text}
variable ... not in scope when solving ...
        \end{minted}
        A metavariable could not be solved due to the solution requiring a
        variable outside of the scope of the metavariable.

  \item \begin{minted}{text}
occurs check failed when solving ...
        \end{minted}
        A metavariable could not be solved due to the solution depending on that
        same metavariable.

  \item \begin{minted}{text}
ambiguous hole due to multiple instances of variable ... in the context
        \end{minted}
        A metavariable could not be solved due to it depending on the same
        variable more than once, so there might not be a unique solution.

  \item \begin{minted}{text}
got nonvariable in the context: ...
        \end{minted}
        A metavariable could not be solved due to it depending on something
        which is not a variable.

  \item \begin{minted}{text}
got non-invertable spine
        \end{minted}
        A metavariable could not be solved due to it being in a spine which is
        not made up of solely function applications.

  \item \begin{minted}{text}
pirec: bug
        \end{minted}
        A bug occurred in the program which is not supposed to happen, an issue
        can be opened in the GitHub repository.
\end{itemize}

\section{Lexical structure}\label{sec:lex}

\caps{EBNF} as specified by \caps{ISO}~\cite{isoebnf} is used to describe the
lexical structure in this section.

Pirec is an indentation sensitive language utilising the off-side rule, like
Haskell~\cite{haskell2010}, Agda~\cite{agda}, and Python~\cite{python}. The
control characters \controlchar{t}, \controlchar{r}, \controlchar{f},
\controlchar{v} and all Unicode space characters are considered white space
characters and are ignored except for delimiting tokens and counting towards
indentation levels. Each space character counts as one indentation level except
for tab (\controlchar{t}) which counts as 8. The newline character
(\controlchar{n}) is also considered white space, but it resets the indentation
level to zero.

Line comments start with \texttt{--} and ends at a newline, block comments are
surrounded by \texttt{\{-} and \texttt{-\}}, they are ignored by the parser.
Block comments can be nested and does not need to be closed if it ends at the
end of the file.

The syntax contains nested \emph{blocks}, each block contains \emph{lines}, the
keywords \mintinline{pirec.py:PirecLexer -x}{let},
\mintinline{pirec.py:PirecLexer -x}{#}, and \mintinline{pirec.py:PirecLexer
  -x}{rec} start a new block. The block can be denoted by indendation, or
alternatively it can be surrounded by braces after the block starting keyword
and lines can be separated by semicolons to have the particular block be
indentation insensitive. To denote a block with indentation, each line inside
the block needs to be more indented than the enclosing block and be at the same
indentation as each other, the end of the block happens before decreased
indentation.
\begin{minted}{ebnf}
block_begin = ? either "{" or increased indentation
                matching with the corresponding block tokens ? ;
block_end   = ? either "}" with optional ";" before it
                or decreased indentation matching
                with the corresponding block tokens ? ;
block_sep   = ? either ";" or the same indentation
                matching with the corresponding block tokens ? ;
\end{minted}

A line in a block can be split across multiple lines and still be parsed as the
same thing, to do this, each token after the first one in the line needs to be
more indented than the enclosing block, and the token at the end of a line break
cannot be a block starting keyword.

Identifiers can consist of any character in the Letter, Mark, Number,
Punctuation, or Symbol category of Unicode, except for some characters in
\caps{ASCII}: parentheses, braces, semicolons, backslashes, periods, and double
quotes.
\begin{minted}{ebnf}
special_char = "(" | ")" | "{" | "}" | ";" | "\" | "." | '"' ;
ident_char   = ? any character not in the Separator
                 or Other category of Unicode ? - special_char ;
\end{minted}
Identifiers must not be keywords, binders can have underscore in place of
identifiers, string literals can be used to denote field labels when it cannot
be written as an identifier.
\begin{minted}{ebnf}
keyword      = "_" | "let" | "=" | ":" | "Type"
             | "→" | "->" | "∀" | "forall" | "λ"
             | "×" | "&" | "∃" | "exists" | "," | "proj1" | "proj2"
             | "Row" | "#" | "|" | "Rec" | "rec" | ":=" ;
ident        = { ident_char } - keyword ;
binder_ident = ident | "_" ;
field_label  = ident | ? string literal enclosed with " ? ;
\end{minted}
Each Unicode token has an \caps{ASCII} alternative which can be used instead
(see \cref{tab:ascii}).
\begin{minted}{ebnf}
arrow  = "→" | "->" ;
forall = "∀" | "forall" ;
lambda = "λ" | "\" ;
times  = "×" | "&" ;
exists = "∃" | "exists" ;
\end{minted}

\section{Syntax}\label{sec:syntax}

The program as a whole is a let block without the starting
\mintinline{pirec.py:PirecLexer -x}{let} token, which will be describe further
below.
\begin{minted}{ebnf}
program = let_block ;
\end{minted}

Expressions can consist of many operators with differing precendence and fixity
(see \cref{tab:ops}), function application can be considered as an operator
denoted by juxtaposition, the argument can be denoted as implicit with braces.
The keywords \mintinline{pirec.py:PirecLexer -x}{proj1},
\mintinline{pirec.py:PirecLexer -x}{proj2}, \mintinline{pirec.py:PirecLexer
  -x}{Row}, and \mintinline{pirec.py:PirecLexer -x}{Rec} act like functions that
must be applied to an argument.
\begin{minted}{ebnf}
expr       = times_expr | times_expr , arrow , expr ;
times_expr = comma_expr | comma_expr , times , times_expr ;
comma_expr = app_expr | app_expr , "," , comma_expr ;
app_expr   = proj_expr | app_expr , proj_expr | app_expr , "{" proj_expr "}"
           | ( "proj1" | "proj2" | "Row" | "Rec" ) , proj_expr ;
proj_expr  = atom | proj_expr , ( "." | ".-" ) , field_label ;
\end{minted}

\begin{table}
  \centering
  \begin{tabular}{@{}cc@{}}
    \toprule
    Operator & Fixity \\\midrule
    \mintinline{pirec.py:PirecLexer -x}{.} and
    \mintinline{pirec.py:PirecLexer -x}{.-}
             & Left
    \\
    function application
             & Left
    \\
    \mintinline{pirec.py:PirecLexer -x}{,}
             & Right
    \\
    \mintinline{pirec.py:PirecLexer -x}{×}
             & Right
    \\
    \mintinline{pirec.py:PirecLexer -x}{→}
             & Right
    \\\bottomrule
  \end{tabular}
  \caption[Pirec operators and their fixity sorted by their precedence]{Pirec
    operators and their fixity ordered by their precedence (from highest to
    lowest precedence)}\label{tab:ops}
\end{table}

The atomic expressions starting with \mintinline{pirec.py:PirecLexer -x}{let},
\mintinline{pirec.py:PirecLexer -x}{∀}, \mintinline{pirec.py:PirecLexer -x}{λ},
\mintinline{pirec.py:PirecLexer -x}{∃}, \mintinline{pirec.py:PirecLexer -x}{#},
or \mintinline{pirec.py:PirecLexer -x}{rec} when they do not have a delimiter at
the right side can lead to ambiguous grammar, hence there is an extra rule which
says that those expressions extend as far to the right as possible. The same
method is used in \caps{GHC}'s BlockArguments language extension~\cite{ghc,
  blockargs}.
\begin{minted}{ebnf}
atom = "(" , expr , ")" | ident | "_" | let_expr | "Type"
     | pi_expr | lam_expr | sigma_expr | row_expr | rec_expr ;
\end{minted}

Let expressions start with a \mintinline{pirec.py:PirecLexer -x}{let} keyword
and it starts a \emph{let block}. Each line of the block is a definition with
optional types, the last line is a single term which is the result.
\begin{minted}{ebnf}
let_expr  = "let" , let_block ;
let_block = block_begin , { let_def , block_sep } , expr , block_end ;
let_def   = binder_ident , [ ":" , expr ] , "=" , expr ;
\end{minted}

The syntax of dependent function types and lambda abstractions are similar, the
only difference is the starting keyword. They can have multiple parameter
binders, each one is an identifier, or it can be surrounded with parentheses
then given a type, or braces to denote it as an implict argument.
\begin{minted}{ebnf}
pi_expr      = forall , fun_binder , { fun_binder } , arrow , expr ;
lam_expr     = lambda , fun_binder , { fun_binder } , arrow , expr ;
fun_binder   = binder_ident | "(" , binder_ident , [ ":" , expr ] , ")"
                            | "{" , binder_ident , [ ":" , expr ] , "}" ;
\end{minted}

Dependent pairs are similar in syntax to dependent functions, except for using
\mintinline{pirec.py:PirecLexer -x}{∃} and \mintinline{pirec.py:PirecLexer
  -x}{×}, and not allowing implicit binders.
\begin{minted}{ebnf}
sigma_expr = exists , sigma_binder , { sigma_binder } , times , times_expr ;
sigma_binder = binder_ident | "(" , binder_ident , [ ":" , expr ] , ")" ;
\end{minted}

Rows literals start with a hash character, and starts a block. The block can be
empty, or it can contain fields, which can be an extension to another row
denoted by a pipe character.
\begin{minted}{ebnf}
row_expr   = "#" , block_begin , [ row_fields ] , block_end ;
row_fields = { row_field , block_sep } , row_field , [ row_ext ] ;
row_field  = field_label , ":" , expr ;
row_ext    = [ block_sep ] , "|" , expr ;
\end{minted}

Record literals have a similar structure to row literals, the extension is the
same. However, record fields are defined with the equals sign, or with
\mintinline{pirec.py:PirecLexer -x}{:=} for updating the content of a field,
they can also be annotated with optional types. Besides these, a record field
can just be a single identifier, which is syntactic sugar for defining it to be
a variable with the same name.
\begin{minted}{ebnf}
rec_expr   = "rec" , block_begin , [ rec_fields ] , block_end ;
rec_fields = { rec_field , block_sep } , rec_field , [ row_ext ] ;
rec_field  = ident | field_label , [ ":" , expr ] , ( "=" | ":=" ) , expr ;
\end{minted}

Some of the syntax is syntactic sugar, which will be desugared after parsing to
get a simpler \caps{AST} (see \cref{fig:sugar,fig:term}, optional typing and
implicit arguments are not shown but they work analogously to their
counterparts).

\begin{figure}
  \begin{align*}
    \Let \{ x_1 = t_1;\ \dots;\ x_n = t_n;\ u \}
     & \quad\mapsto\quad
    \Let \{ x_1 = t_1;\ \dots \Let \{ x_n = t_n;\ u \} \dots \}
    \\
    \forall (x_1 : A_1) \dots (x_n : A_n) \to B
     & \quad\mapsto\quad
    \forall (x_1 : A_1) \to \dots \forall (x_n : A_n) \to B
    \\
    \lambda x_1 \dots x_n \to B
     & \quad\mapsto\quad
    \lambda x_1 \to \dots \lambda x_n \to B
    \\
    \exists (x_1 : A_1) \dots (x_n : A_n) \times B
     & \quad\mapsto\quad
    \exists (x_1 : A_1) \times \dots \forall (x_n : A_n) \times B
    \\
    \# \{ l_1 : t_1;\ \dots;\ l_n : t_n \}
     & \quad\mapsto\quad
    \# \{ l_1 : t_1 \mid \dots \# \{ l_n : t_n \mid \# \{ \} \} \dots \}
    \\
    \# \{ l_1 : t_1;\ \dots;\ l_n : t_n \mid r \}
     & \quad\mapsto\quad
    \# \{ l_1 : t_1 \mid \dots \# \{ l_n : t_n \mid r \} \dots \}
    \\
    \RecordLit \{ l_1 = t_1;\ \dots;\ l_n = t_n \}
     & \quad\mapsto\quad
    \RecordLit \{ l_1 = t_1 \mid \dots
    \RecordLit \{ l_n = t_n \mid \RecordLit \{ \} \} \dots \}
    \\
    \RecordLit \{ l_1 = t_1;\ \dots;\ l_n = t_n \mid u \}
     & \quad\mapsto\quad
    \RecordLit \{ l_1 = t_1 \mid \dots
    \RecordLit \{ l_n = t_n \mid u \} \dots \}
    \\
    \RecordLit \{ l \coleqq t \mid u \}
     & \quad\mapsto\quad
    \RecordLit \{ l \coleqq t \mid u \RecordRestr l \}
    \\
    \RecordLit \{ l \mid u \}
     & \quad\mapsto\quad
    \RecordLit \{ l = l \mid u \}
  \end{align*}
  \caption{Syntactic sugar in Pirec}\label{fig:sugar}
\end{figure}

\begin{figure}
  \begin{alignat*}{2}
    t, u, v, w, r, A, B, R \Coleqq{}
     & x
     & \qquad\text{variable}
    \\
    \mid{}
     & \_
     & \qquad\text{hole}
    \\
    \mid{}
     & {\Let \{ x = t;\ u \}}
     & \qquad\text{let expression}
    \\
    \mid{}
     & {\Univ}
     & \qquad\text{universe}
    \\
    \mid{}
     & \forall (x : A) \to B
     & \qquad\text{dependent function}
    \\
    \mid{}
     & \lambda x \to t
     & \qquad\text{lambda abstraction}
    \\
    \mid{}
     & t\ u
     & \qquad\text{function application}
    \\
    \mid{}
     & \exists (x : A) \times B
     & \qquad\text{dependent pair}
    \\
    \mid{}
     & t, u
     & \qquad\text{pair constructor}
    \\
    \mid{}
     & {\ProjOne t}
     & \qquad\text{first projection}
    \\
    \mid{}
     & {\ProjTwo t}
     & \qquad\text{second projection}
    \\
    \mid{}
     & {\RowType A}
     & \qquad\text{row type}
    \\
    \mid{}
     & \# \{ \}
     & \qquad\text{empty row}
    \\
    \mid{}
     & \# \{ l : t \mid r \}
     & \qquad\text{row extension}
    \\
    \mid{}
     & {\RecordType R}
     & \qquad\text{record type}
    \\
    \mid{}
     & {\RecordLit \{ \}}
     & \qquad\text{empty record}
    \\
    \mid{}
     & {\RecordLit \{ l = t \mid u \}}
     & \qquad\text{record extension}
    \\
    \mid{}
     & t.l
     & \qquad\text{record projection}
    \\
    \mid{}
     & t \RecordRestr l
     & \qquad\text{record restriction}
  \end{alignat*}
  \caption{The desugared terms of Pirec}\label{fig:term}
\end{figure}

\section{Type system and Semantics}\label{sec:typesem}

The typing rules of Pirec are presented in \cref{fig:typing}, some equalities in
for typing are shown in \cref{fig:equality}, holes and unannotated binders will
have metavariables inserted, then the program will try to solve them.

\begin{figure}
  \begin{gather*}
    \begin{prooftree}
      \infer0{\Gamma,\ x : A \vdash x : A}
    \end{prooftree}
    \qquad
    \begin{prooftree}
      \hypo{\Gamma \vdash t : A}
      \hypo{\Gamma \vdash u : B[x \coleqq t]}
      \infer2{\Gamma
        \vdash \Let \{ x = t;\ u \} : B}
    \end{prooftree}
    \\
    \begin{prooftree}
      \infer0{\Gamma \vdash \Univ : \Univ}
    \end{prooftree}
    \qquad
    \begin{prooftree}
      \hypo{\Gamma \vdash A : \Univ}
      \hypo{\Gamma,\ x : A \vdash B : \Univ}
      \infer2{\Gamma \vdash \forall (x : A) \to B : \Univ}
    \end{prooftree}
    \\
    \begin{prooftree}
      \hypo{\Gamma,\ x : A \vdash t : B}
      \infer1{\Gamma \vdash \lambda x \to t : \forall (x : A) \to B}
    \end{prooftree}
    \qquad
    \begin{prooftree}
      \hypo{\Gamma \vdash t : \forall (x : A) \to B}
      \hypo{\Gamma \vdash u : A}
      \infer2{\Gamma \vdash t\ u : B[x \coleqq u]}
    \end{prooftree}
    \\
    \begin{prooftree}
      \hypo{\Gamma \vdash A : \Univ}
      \hypo{\Gamma,\ x : A \vdash B : \Univ}
      \infer2{\Gamma \vdash \exists (x : A) \times B : \Univ}
    \end{prooftree}
    \qquad
    \begin{prooftree}
      \hypo{\Gamma \vdash t : A}
      \hypo{\Gamma \vdash u : B}
      \infer2{\Gamma \vdash t, u : \exists (x : A) \times B}
    \end{prooftree}
    \\
    \begin{prooftree}
      \hypo{\Gamma \vdash A : \Univ}
      \hypo{\Gamma,\ x : A \vdash B : \Univ}
      \infer2{\Gamma \vdash \exists (x : A) \times B : \Univ}
    \end{prooftree}
    \qquad
    \begin{prooftree}
      \hypo{\Gamma \vdash t : A}
      \hypo{\Gamma \vdash u : B[x \coleqq t]}
      \infer2{\Gamma \vdash t, u : \exists (x : A) \times B}
    \end{prooftree}
    \\
    \begin{prooftree}
      \hypo{\Gamma \vdash t : \exists (x : A) \times B}
      \infer1{\Gamma \vdash \ProjOne t : A}
    \end{prooftree}
    \qquad
    \begin{prooftree}
      \hypo{\Gamma \vdash t : \exists (x : A) \times B}
      \infer1{\Gamma \vdash \ProjTwo t : B[x \coleqq \ProjOne t]}
    \end{prooftree}
    \\
    \begin{prooftree}
      \hypo{\Gamma \vdash A : \Univ}
      \infer1{\Gamma \vdash \RowType A : \Univ}
    \end{prooftree}
    \qquad
    \begin{prooftree}
      \infer0{\Gamma \vdash \# \{ \} : \RowType A}
    \end{prooftree}
    \qquad
    \begin{prooftree}
      \hypo{\Gamma \vdash t : A}
      \hypo{\Gamma \vdash r : \RowType A}
      \infer2{\Gamma \vdash \# \{ l : t \mid r \} : \RowType A}
    \end{prooftree}
    \\
    \begin{prooftree}
      \hypo{\Gamma \vdash R : \RowType \Univ}
      \infer1{\Gamma \vdash \RecordType R : \Univ}
    \end{prooftree}
    \qquad
    \begin{prooftree}
      \infer0{\Gamma \vdash \RecordLit \{ \} : \RecordType \# \{ \}}
    \end{prooftree}
    \qquad
    \begin{prooftree}
      \hypo{\Gamma \vdash t : A}
      \hypo{\Gamma \vdash u : \RecordType R}
      \infer2{\Gamma \vdash \RecordLit \{ l = t \mid u \}
        : \RecordType \# \{ l : A \mid R \}}
    \end{prooftree}
    \\
    \begin{prooftree}
      \hypo{\Gamma \vdash t : \RecordType \# \{ l : A \mid R \}}
      \infer1{\Gamma \vdash t.l : A}
    \end{prooftree}
    \qquad
    \begin{prooftree}
      \hypo{\Gamma \vdash t : \RecordType \# \{ l : A \mid R \}}
      \infer1{\Gamma \vdash t \RecordRestr l : \RecordType R}
    \end{prooftree}
  \end{gather*}
  \caption{The typing rules of Pirec}\label{fig:typing}
\end{figure}

\begin{figure}
  \begin{gather*}
    \begin{prooftree}
      \infer0{(\lambda x \to t)\ u \equiv t[x \coleqq u]}
    \end{prooftree}
    \qquad
    \begin{prooftree}
      \infer0{(\lambda x \to t\ x) \equiv t}
    \end{prooftree}
    \\
    \begin{prooftree}
      \infer0{\ProjOne (t, u) \equiv t}
    \end{prooftree}
    \qquad
    \begin{prooftree}
      \infer0{\ProjTwo (t, u) \equiv u}
    \end{prooftree}
    \qquad
    \begin{prooftree}
      \infer0{(\ProjOne t, \ProjTwo t) \equiv t}
    \end{prooftree}
    \\
    \begin{prooftree}
      \hypo{l \neq l'}
      \infer1{\# \{ l = t \mid \# \{ l' = u \mid r \} \}
        \equiv \# \{ l' = u \mid \# \{ l = t \mid r \} \}}
    \end{prooftree}
    \\
    \begin{prooftree}
      \hypo{l \neq l'}
      \infer1{\RecordLit \{ l = t \mid \RecordLit \{ l' = u \mid v \} \}
        \equiv \RecordLit \{ l' = u \mid \RecordLit \{ l = t \mid v \} \}}
    \end{prooftree}
    \\
    \begin{prooftree}
      \infer0{\RecordLit \{ l = t \mid u \}.l \equiv t}
    \end{prooftree}
    \qquad
    \begin{prooftree}
      \hypo{l \neq l'}
      \infer1{\RecordLit \{ l' = t \mid u \}.l \equiv u.l}
    \end{prooftree}
    \qquad
    \begin{prooftree}
      \infer0{\RecordLit \{ l_1 = t.l_1; \dots; l_n = t.l_n \} \equiv t}
    \end{prooftree}
    \\
    \begin{prooftree}
      \infer0{\RecordLit \{ l = t \mid u \} \RecordRestr l \equiv u }
    \end{prooftree}
    \qquad
    \begin{prooftree}
      \hypo{l \neq l'}
      \infer1{\RecordLit \{ l' = t \mid u \} \RecordRestr l \equiv
        \RecordLit \{ l' = t \mid u \RecordRestr l \}}
    \end{prooftree}
  \end{gather*}
  \caption{Non-trivial equality rules in Pirec}\label{fig:equality}
\end{figure}

The row polymorphism and extensible record implementation is based
on~\cite{scopedlabels}, so rows and records can contain multiple fields, and the
most recently extended field \enquote{shadows} but does not remove fields with
the same name, shadowed fields are still accessible by removing the fields
\enquote{on top} of them. Fields with different labels are still unordered
relative to one another.

The evaluation requires the definition of neutral terms, which are terms with a
variable and actions on that variable that cannot proceed due to it being a
variable. It is shown in \cref{fig:neutral}. The big-step operational semantics
of evaluating Pirec terms to normal form is defined in \Cref{fig:big-step}.

\begin{figure}
  \begin{alignat*}{2}
    n \Coleqq{} & x
                & \qquad\text{variable}             \\
    \mid{}      & n\ t
                & \qquad\text{function application} \\
    \mid{}      & {\ProjOne n}
                & \qquad\text{first projection}     \\
    \mid{}      & {\ProjTwo n}
                & \qquad\text{second projection}    \\
    \mid{}      & \# \{ l : t \mid n \}
                & \qquad\text{row extension}        \\
    \mid{}      & {\RecordLit \{ l = t \mid n \}}
                & \qquad\text{record extension}     \\
    \mid{}      & n.l
                & \qquad\text{record projection}    \\
    \mid{}      & n \RecordRestr l
                & \qquad\text{record restriction}
  \end{alignat*}
  \caption{The neutral terms of Pirec}\label{fig:neutral}
\end{figure}

\begin{figure}
  \begin{gather*}
    \begin{prooftree}
      \infer0{x \Downarrow x}
    \end{prooftree}
    \qquad
    \begin{prooftree}
      \hypo{u[x \coleqq t] \Downarrow u'}
      \infer1{\Let \{ x = t;\ u \} \Downarrow u'}
    \end{prooftree}
    \\
    \begin{prooftree}
      \infer0{\Univ \Downarrow \Univ}
    \end{prooftree}
    \qquad
    \begin{prooftree}
      \hypo{A \Downarrow A'}
      \hypo{B \Downarrow B'}
      \infer2{\forall (x : A) \to B \Downarrow \forall (x : A') \to B'}
    \end{prooftree}
    \\
    \begin{prooftree}
      \hypo{t \Downarrow t'}
      \infer1{\lambda x \to t \Downarrow \lambda x \to t'}
    \end{prooftree}
    \qquad
    \begin{prooftree}
      \hypo{t \Downarrow \lambda x \to v}
      \hypo{v[x \coleqq u] \Downarrow v'}
      \infer2{t\ u \Downarrow v'}
    \end{prooftree}
    \qquad
    \begin{prooftree}
      \hypo{t \Downarrow n}
      \hypo{u \Downarrow u'}
      \infer2{t\ u \Downarrow n\ u'}
    \end{prooftree}
    \\
    \begin{prooftree}
      \hypo{A \Downarrow A'}
      \hypo{B \Downarrow B'}
      \infer2{\exists (x : A) \times B \Downarrow \exists (x : A') \times B'}
    \end{prooftree}
    \qquad
    \begin{prooftree}
      \hypo{t \Downarrow t'}
      \hypo{u \Downarrow u'}
      \infer2{t, u \Downarrow t', u'}
    \end{prooftree}
    \\
    \begin{prooftree}
      \hypo{t \Downarrow u, v}
      \infer1{\ProjOne t \Downarrow u}
    \end{prooftree}
    \qquad
    \begin{prooftree}
      \hypo{t \Downarrow n}
      \infer1{\ProjOne t \Downarrow \ProjOne n}
    \end{prooftree}
    \qquad
    \begin{prooftree}
      \hypo{t \Downarrow u, v}
      \infer1{\ProjTwo t \Downarrow v}
    \end{prooftree}
    \qquad
    \begin{prooftree}
      \hypo{t \Downarrow n}
      \infer1{\ProjTwo t \Downarrow \ProjTwo n}
    \end{prooftree}
    \\
    \begin{prooftree}
      \hypo{A \Downarrow A'}
      \infer1{\RowType A \Downarrow \RowType A'}
    \end{prooftree}
    \qquad
    \begin{prooftree}
      \infer0{\# \{ \} \Downarrow \# \{ \}}
    \end{prooftree}
    \qquad
    \begin{prooftree}
      \hypo{t \Downarrow t'}
      \hypo{r \Downarrow r'}
      \infer2{\# \{ l : t \mid r \} \Downarrow \# \{ l : t' \mid r' \}}
    \end{prooftree}
    \\
    \begin{prooftree}
      \hypo{R \Downarrow R'}
      \infer1{\RecordType R \Downarrow \RecordType R'}
    \end{prooftree}
    \qquad
    \begin{prooftree}
      \infer0{\RecordLit \{ \} \Downarrow \RecordLit \{ \}}
    \end{prooftree}
    \qquad
    \begin{prooftree}
      \hypo{t \Downarrow t'}
      \hypo{u \Downarrow u'}
      \infer2{\RecordLit \{ l = t \mid u \}
        \Downarrow \RecordLit \{ l = t' \mid u' \}}
    \end{prooftree}
    \\
    \begin{prooftree}
      \hypo{t \Downarrow \RecordLit \{ l = u \mid v \}}
      \infer1{t.l \Downarrow u}
    \end{prooftree}
    \qquad
    \begin{prooftree}
      \hypo{t \Downarrow \RecordLit \{ l' = u \mid v \}}
      \hypo{l \neq l'}
      \hypo{v.l \Downarrow w}
      \infer3{t.l \Downarrow w}
    \end{prooftree}
    \qquad
    \begin{prooftree}
      \hypo{t \Downarrow n}
      \infer1{t.l \Downarrow n.l}
    \end{prooftree}
    \\
    \begin{prooftree}
      \hypo{t \Downarrow \RecordLit \{ l = u \mid v \}}
      \infer1{t \RecordRestr l \Downarrow v}
    \end{prooftree}
    \qquad
    \begin{prooftree}
      \hypo{t \Downarrow \RecordLit \{ l' = u \mid v \}}
      \hypo{l \neq l'}
      \hypo{v \RecordRestr l \Downarrow v'}
      \infer3{t \RecordRestr l \Downarrow \RecordLit \{ l' = u \mid v' \}}
    \end{prooftree}
    \qquad
    \begin{prooftree}
      \hypo{t \Downarrow n}
      \infer1{t \RecordRestr l \Downarrow n \RecordRestr l}
    \end{prooftree}
  \end{gather*}
  \caption{The big-step operational semantics of Pirec}\label{fig:big-step}
\end{figure}
